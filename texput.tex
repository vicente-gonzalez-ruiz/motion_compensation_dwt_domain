% Emacs, this is -*-latex-*-

\title{\href{https://vicente-gonzalez-ruiz.github.io/motion_estimation/docs/}{Motion Compensation in the Discrete Wavelet Transform Domain}}

\maketitle

\section{Motion estimation in the critically sampled DWT domain}
%{{{

\begin{figure}
  \centering
  \begin{tabular}{ccc}
    \vbox{\png{frame_0_Y}{300}} & \vbox{\png{frame_1_Y}{300}} & \vbox{\png{frame_2_Y}{300}} \\
    & \vbox{\svg{movement_0}{300}} & \vbox{\svg{movement_1}{300}} \\
    \vbox{\png{f0_haar_LL}{300}} & \vbox{\png{f1_haar_LL}{300}} & \vbox{\png{f2_haar_LL}{300}} \\
    \vbox{\png{f0_haar_LH}{300}} & \vbox{\png{f1_haar_LH}{300}} & \vbox{\png{f2_haar_LH}{300}} \\
    \vbox{\png{f0_haar_HL}{300}} & \vbox{\png{f1_haar_HL}{300}} & \vbox{\png{f2_haar_HL}{300}} \\
    \vbox{\png{f0_haar_HH}{300}} & \vbox{\png{f1_haar_HH}{300}} & \vbox{\png{f2_haar_HH}{300}} \\
    & \vbox{\svg{f0_1_haar_LL}{300}} & \vbox{\svg{f0_2_haar_LL}{300}} \\
    & \vbox{\svg{f0_1_haar_LH}{300}} & \vbox{\svg{f0_2_haar_LH}{300}} \\
    & \vbox{\svg{f0_1_haar_HL}{300}} & \vbox{\svg{f0_2_haar_HL}{300}} \\
    & \vbox{\svg{f0_1_haar_HH}{300}} & \vbox{\svg{f0_2_haar_HH}{300}}
  \end{tabular}
  \caption{A demonstration of the shift-variance of the DWT. As it can
    be seen, when the circle has been moved only one pixel, the value
    of the coefficients that correspond to the circunference of the
    circle are different between the reference frame and the predicted
    frame. Similar results have been obtained for other filters. See
    the notebook
    \href{https://github.com/vicente-gonzalez-ruiz/motion_compensation_dwt_domain/blob/main/DWT_shift_invariance.ipynb}{Shift
      invariance in the DWT domain}.}
  \label{fig:dwt_shift_variance}
\end{figure}

This technique is also known by In-Band Motion estimation and
Compensation (IBMC)~\cite{andreopoulos2005complete}. ME in the DWT is
interesting because the multiresolution structure generated by the DWT
can speed-up the computation of the motion information because,
potentially, we can use hierarchical ME directly over the DWT
domain. However, the DWT, that is critically sampled, is not shift
invariant (see the Fig.~\ref{fig:dwt_shift_variance}). This basically
means that the same signal displaced (or delayed) one sample generate
different DWT coefficients. Therefore, we cannot estimate nor
compensate the motion in the DWT domain simply comparing blocks of the
signal.

Notice also that the effects of shift-variance is also visible
after using the inverse transform when the coefficients are filtered
or quantized, because the aliasing between the filters is not
completely cancelled in this case~\cite{bradley2003shift}.

The reason why the 1-pixel movement is generating different
coefficients in the reference and the predicted frames is because a
1-pixel motion cannot be represented using always the same phase
(remember that with the downsampler we are basically selecting only
one the two possible phases of the output of the analysis filters: the
even samples or the odd samples). Lets suppose that the downsampler
discards the odd coefficients (let's refer them as odd-phase
coefficients). In this case, the even-phase cofficients of the
reference frame are the same than the odd-phase coefficients of the
predicted frame (this can be seen in this notebook). Therefore, in the
1D case, when the motion is ``even''-type (that is, a displacement of
a even number of samples) we should compensate the even-phase
coefficients of the reference and the predicted frame, while when the
motion is ``odd''-type we should compensate the odd-phase coefficients
of the predicted frame with a prediction generated with the even-phase
coefficients of the reference frame, or viceversa.

There are different alternatives for recovering the ``lost'' phase
during the DWT (in the 1D case):
\begin{enumerate}
\item MC-then-downsample: Perform first the MC stage directly over the
  output of the analysis filters, and then, selectively downsample the
  result. Notice that the downsampler should select the right phase,
  depending on the type of motion detected (``odd'' or ``even''). This
  information (the selected phase), should be available at the
  decoder, along with the motion fields.
\item Delay-then-DWT: Perform two identical DWTs, one to the original
  signal, and the other to a one-sample delayed signal (remember than
  a movement of one pixel will change the phase at the output of the
  DWT). Thus, the DWT applied to the original signal will generate
  one of the phases and the DWT applied to the delayed signal will
  generate the other one.
\item CODWT: Use the current (single phase) L and H coefficients to
  compute the missing phase, using the CODWT (Complete-to-Overcomplete
  DWT)~\cite{andreopoulos2005complete} (a new type of DWT applied to
  the DWT coefficients).
\item Use the Algorithme \`a Trous (AaT)~\cite{mallat1999wavelet},
  which removes the downsamplers from the DWT, generating the so
  called Overcomplete DWT (ODWT). Notice that, because the
  downsamplers are removed, the aliasing artifacts produced by the
  downsamplers is also avoided.
\item Approximate the AaT coefficients by interpolating the DWT
  coefficients using the DWT synthesis filters. In this case, the
  aliasing is not avoided, but the shift-variance problem is
  reduced.
\end{enumerate}
Each alternative has pros and cons. If the DWT has been implemented
using convolution, MC-then-downsample should be a fast
alternative. However, if the DWT uses Lifting, Multiple-DWT-then-MC
should be fast also, because only one phase is computed by the
DWT. These two options can be used with any DWT filters. On the other
hand, CODWT needs specific designs form each DWT filters. Notice that,
in any case, the solution is reached after using the ODWT domain.

In the 2D case, and always working with only one level of the DWT, we
have up to four different phases: (even, even)-, (even, odd)-, (odd,
even)-, and (odd, odd)-phase coefficients. Thus, depending on the type
of motion detected, the corresponding phase should be selected.

\begin{comment}
However, suprisingly, at it can be also seen in the
Fig.~\ref{fig:DWT}, when the circle has traveled two pixels (frames 0
and 2), a perfect match is achieved! The reason why the 1-pixel motion
generates different coefficients in the reference and the predicted
frames, and the same coefficients for a 2-pixel motion is because, in
the first case the right coefficients were discarded by the
downsamplers, and in the second case not.

Usually, we call \emph{phases} to the two possible coefficients
resulting from one (1D) filter to be subsampled, being the even phase,
the even coefficients, and the odd phase, the odd
coefficients. Therefore, when the motion is of type ``even'' (when we
have a $2N$-pixels motion), we should use the even phase to compensate
the frames, and viceversa (use the odd phase to compensate a
$2N+1$-pixels motion). Notice that in the 2D case, and always working
with only one level of the DWT, we have up to four different phases:
(even, even)-, (even, odd)-, (odd, even)-, and (odd, odd)-phase
coefficients. Thus, depending on the type of motion detected, the
corresponding phase should be selected.
\end{comment}

\subsection{Recovering the lost phases}

\begin{figure}
  \centering
  \begin{tabular}{ccc}
    \vbox{\png{f0_ohaar_LL}{300}} & \vbox{\png{f1_ohaar_LL}{300}} & \vbox{\png{f2_ohaar_LL}{300}} \\
    \vbox{\png{f0_ohaar_LH}{300}} & \vbox{\png{f1_ohaar_LH}{300}} & \vbox{\png{f2_ohaar_LH}{300}} \\
    \vbox{\png{f0_ohaar_HL}{300}} & \vbox{\png{f1_ohaar_HL}{300}} & \vbox{\png{f2_ohaar_HL}{300}} \\
    \vbox{\png{f0_ohaar_HH}{300}} & \vbox{\png{f1_ohaar_HH}{300}} & \vbox{\png{f2_ohaar_HH}{300}} \\
    & \vbox{\svg{f0_1_ohaar_LL}{300}} & \vbox{\svg{f0_2_ohaar_LL}{300}} \\
    & \vbox{\svg{f0_1_ohaar_LH}{300}} & \vbox{\svg{f0_2_ohaar_LH}{300}} \\
    & \vbox{\svg{f0_1_ohaar_HL}{300}} & \vbox{\svg{f0_2_ohaar_HL}{300}} \\
    & \vbox{\svg{f0_1_ohaar_HH}{300}} & \vbox{\svg{f0_2_ohaar_HH}{300}}
  \end{tabular}
  \caption{A demonstration of the shift-invariance of the ODWT. See the notebook
    \href{https://github.com/vicente-gonzalez-ruiz/motion_compensation_dwt_domain/blob/main/ODWT_shift_invariance.ipynb}{Shift invariance in the Overcomplete DWT domain}.}
\label{fig:odwt}
\end{figure}

There are different alternatives for regenerating the phases discarded
by the subsamplers of the DWT. This is equivalent to compute the
Overcomplete DWT (ODWT)~\cite{mallat1999wavelet}.
\begin{enumerate}
\item Use the Algorithme \`a Trous~\cite{mallat1999wavelet}, which
  basically consists in removing the downsamplers, avoiding thus the
  aliasing artifacts generated by the noncompliance with the sampling
  theorem. See this
  \href{https://github.com/Sistemas-Multimedia/Sistemas-Multimedia.github.io/blob/master/milestones/11-MC_in_DWT_domain/regenerating.ipynb}{notebook}.
\item Considering the previous experiments, it's easy to see that if
  we shift the signal one sample and perform the DWT, we get the
  ``lost'' phase. This method has been used to perform efficient MC in
  the DWT domain~\cite{park2000motion,li2001all}. See this
  \href{https://github.com/Sistemas-Multimedia/Sistemas-Multimedia.github.io/blob/master/milestones/11-MC_in_DWT_domain/ODWT_with_delay.ipynb}{notebook}.
\item Apply some transform (such as for example, the CODWT
  (Complete-to-Overcomplete DWT)~\cite{andreopoulos2005complete} to
  the DWT to reconstruct the ODWT.
\end{enumerate}
The Fig.~\ref{fig:odwt} shows the shift invariance of the ODWT.

\subsection{About using the lost phases in IBMC}
Up to date, all the video codecs that use critically sampled IBMC also
use
\href{https://vicente-gonzalez-ruiz.github.io/video_compression/}{block-based
  motion compensation}. This technique divides the frames into
non-overlaping blocks and computes a motion vector for every block,
that provides a projection (a prediction) $\hat{P}$ of the reference
frame $R$ that must be as close as possible to the predicted frame
$P$. These blocks usually have a size of 16x16 pixels.

The use of blocks imples that:
\begin{enumerate}
\item If $N$ is the number of pixels in a frame, $N/256$ (for 16x16
  blocks) is the number of motion vectors. Therefore, if the motion
  vectors field has to be sent to the decoder, the data overhead is
  small (although this depends on the length of the representation of
  the texture).
\item All the coefficients that correspond to the same block has the
  same phase. Thus, if the phase also has to be sent to the decoder,
  again, the data overhead can be considered small.
\end{enumerate}

Unfortunately, there is a problem with mixing the phases. To
reconstruct the border pixels of the blocks, the adjacent (with the
same phase) coefficients must be also used by the decoder (see this
\href{https://github.com/Sistemas-Multimedia/Sistemas-Multimedia.github.io/blob/master/milestones/11-MC_in_DWT_domain/mixing_phases.ipynb}{notebook}). For
this reason, the size of the blocks affects to the compression ratio
(the smaller the blocks, the higher the number of adjacent
coefficients, and therefore, the lower the compression ratio). We can
think that this effect can be mitigated using larger block sizes, but
this will also affect to the compression ratio because the quality of
the predictions worsen with the increment of the size of the
blocks. This carries an optimization problem that it's hard to solve,
especially in real-time applications.

%}}}

\section{References}
%{{{

\renewcommand{\addcontentsline}[3]{}% Remove functionality of \addcontentsline
\bibliography{image_pyramids,DWT,motion_estimation,HEVC}

%}}}

\begin{comment}
\begin{figure}
  \centering
  \begin{tabular}{ccc}
    \vbox{\png{moving_circle_000}{400}} & % create_moving_circles.ipynb
    \vbox{\svg{movement_0}{400}} &
    \vbox{\png{difference_0}{400}} \\
    \vbox{\svg{haar_LL}{400}} & % DWT_shift_invariance.ipynb
    \vbox{\svg{db5_LL}{400}} &
    \vbox{\svg{bior35_LL}{400}} \\
    \vbox{\myfig{haar_LH}{4cm}{400}} &
    \vbox{\myfig{db5_LH}{4cm}{400}} &
    \vbox{\myfig{bior35_LH}{4cm}{400}} \\ 
    \vbox{\myfig{haar_HL}{4cm}{400}} &
    \vbox{\myfig{db5_HL}{4cm}{400}} &
    \vbox{\myfig{bior35_HL}{4cm}{400}} \\
    \vbox{\myfig{haar_HH}{4cm}{400}} &
    \vbox{\myfig{db5_HH}{4cm}{400}} &
    \vbox{\myfig{bior35_HH}{4cm}{400}} 
  \end{tabular}
  \caption{A demonstration of the shift-variance of the DWT.}
\end{figure}
\end{comment}  
