% Emacs, this is -*-latex-*-

\title{href{https://vicente-gonzalez-ruiz.github.io/motion_estimation/docs/}{Motion Compensation in the Discrete Wavelet Transform Domain}

\maketitle

\section{Difficulties}
%{{{

\begin{figure}
  \centering
  \begin{tabular}{ccc}
    \vbox{\pngfig{moving_circle_000}{4cm}{400}} &
    \vbox{\myfig{movement}{4cm}{400}} &
    \vbox{\pngfig{difference_0}{4cm}{400}} \\
    \vbox{\myfig{haar_LL}{4cm}{400}} &
    \vbox{\myfig{db5_LL}{4cm}{400}} &
    \vbox{\myfig{bior35_LL}{4cm}{400}} \\
    \vbox{\myfig{haar_LH}{4cm}{400}} &
    \vbox{\myfig{db5_LH}{4cm}{400}} &
    \vbox{\myfig{bior35_LH}{4cm}{400}} \\ 
    \vbox{\myfig{haar_HL}{4cm}{400}} &
    \vbox{\myfig{db5_HL}{4cm}{400}} &
    \vbox{\myfig{bior35_HL}{4cm}{400}} \\
    \vbox{\myfig{haar_HH}{4cm}{400}} &
    \vbox{\myfig{db5_HH}{4cm}{400}} &
    \vbox{\myfig{bior35_HH}{4cm}{400}} 
  \end{tabular}
  \caption{A demonstration of the shift-variance of the DWT.}
\label{fig:dwt_shift_variance}
\end{figure}

This technique is also known by In-Band Motion estimation and
Compensation (IBMC)~\cite{andreopoulos2005complete}. ME in the DWT is
interesting because the multiresolution structure generated by the DWT
can speed-up the computation of the motion information because,
potentially, we can use hierarchical ME directly over the DWT
domain. However, the DWT, that is critically sampled, is not shift
invariant (see the Fig.~\ref{fig:dwt_shift_variance}). This basically
means that the same signal displaced (or delayed) one sample generate
different DWT coefficients. Therefore, we cannot estimate the motion
in the DWT domain simply comparing blocks of the signal.

Solutions:
\begin{enumerate}
\item Recompute the decimated coefficients from the existing ones,
  generating the Overcomplete DWT.
\item Compute the inverse DWT and estimate in the image domain.
\end{enumerate}

%}}}

\section{References}
%{{{

\renewcommand{\addcontentsline}[3]{}% Remove functionality of \addcontentsline
\bibliography{image_pyramids,DWT,motion_estimation,HEVC}

%}}}
